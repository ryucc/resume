%%%%%%%%%%%%%%%%%%%%%%%%%%%%%%%%%%%%%%%%%
% Medium Length Professional CV 
% LaTeX Template
% Version 2.0 (8/5/13)
%
% This template has been downloaded from:
% http://www.LaTeXTemplates.com
%
% Original author:
% Trey Hunner (http://www.treyhunner.com/)
%
% Important note:
% This template requires the resume.cls file to be in the same directory as the
% .tex file. The resume.cls file provides the resume style used for structuring the
% document.
%
%%%%%%%%%%%%%%%%%%%%%%%%%%%%%%%%%%%%%%%%%

%----------------------------------------------------------------------------------------
%	PACKAGES AND OTHER DOCUMENT CONFIGURATIONS
%----------------------------------------------------------------------------------------

\documentclass{resume} % Use the custom resume.cls style

\usepackage[left=0.75in,top=0.1in,right=0.75in,bottom=0.1in]{geometry} % Document margins

%\address{130 N.Orchard St. Unit #3, Madison, WI}
%\address{(608) 421-4190}
\begin{document}
\begin{center}
{\MakeUppercase{\bf Kurtis Liu}}\\
US Citizen.
(608) 421-4190, liukurtis@gmail.com
\end{center}
%----------------------------------------------------------------------------------------
%	OBJECTIVE SECTION
%----------------------------------------------------------------------------------------

%\begin{rSection}{objective}
%Looking a Software Engineering Position, January %2018.
%\end{rSection}

%------------------------------------------------------------------
% ---------------------
%	EDUCATION SECTION
%----------------------------------------------------------------------------------------
\begin{rSection}{Education}
{\bf University of Wisconsin - Madison} \hfill {09/2016 - 12/2017 (expected)}\\ 
M.S. in {\bf Computer Science} \\
Selected courses:  Machine Learning, Computer Vision, Image Processing, Pattern Recognition


{\bf National Taiwan University} \hfill   {09/2012 - 06/2016}\\ 
B.S. in Engineering,  {\bf Computer Science} and Information Engineering\\
Minor, {\bf Mathematics} \\
Selected courses: Functional Analysis, Complex Analysis, Probability\\
%Video Encoding and Communication, Computer Security, Parallel Programming\\
%2013/01 Dean's Award for 1st in class.\\
%2014/09 Dean's Award for 6th in class.
GPA: 3.84
\end{rSection}
%------------------------------------------------
%----------------------------------------------------------------------------------------
%	WORK EXPERIENCE SECTION
%----------------------------------------------------------------------------------------
\begin{rSection}{Work Experience}

    \begin{rSubsection}{Accuray Inc. -- Software Engineer Intern}{05/2017 - 08/2017}{}{}%{\bf Software Engineer Intern }{Supervisor: Gabe Westmont}
%\item Design embedded system core utilities with C++.
%\item Design user interfaces with C++/wxWidgets.
%\item Designed and developed scheduling utilities.
%\item Developed Python interfaces for C++ components.
\item Embedded software design for the Radixact radiation therapy system.
\item Designed internal scheduler on Linux, using C++11/Boost.
\item Developed Messaging System, GUIs, Python interfaces to C++ controlled hardware. 
%\item Automate testing with python/bash scripting.
\end{rSubsection}

\begin{rSubsection}{NTU Lab for Language Technologies -- Software Developer}{09/2014 - 10/2014}{}{}%{Supervisor: Prof.Zhao-Ming Gao}
\item Statistical Natural Language Processing for English Essay Grading.
\item Feature extraction, Linear regression, and Hypothesis testing.
\item Connected the R/Python/PHP Codebase with shell scripts.
\item Optimized previously O($n^2\log^2n$) queries to O($n\log n$).
\end{rSubsection}

\begin{rSubsection}{UW-Madison - Teaching Assistant}{08/2016 - Present}{}{}%{\bf Short-Term Software Contractor}{}
\item CS540 -- Artificial Intellegence, CS240 -- Discrete Mathematics, CS310 -- Problem Solving for Engineers
\item Searching algorithms, baysian models, Supervised learning
\item Sets, Proofs, Program Correctness, Recurrences, Asymptotic Analysis, Finite Automata
\item Matlab programming, Numerical methods
\end{rSubsection}

%\begin{rSubsection}{Lab of Algorithmic Research}{January 2014 - June 2016}{Undergraduate Researcher}{Supervisor: Prof.Hsueh-I Lu}
%\item Multiple s-t pair Non-crossing Shortest Paths
%\end{rSubsection}
%\begin{rSubsection}{Cavedu Intelligent House - Short-Term Software Contractor}{06/2016}{}{}%{\bf Short-Term Software Contractor}{}
%\item Built face recognition system, optimize OpenCV code for Raspberry PI.
%\item Github: https://github.com/jerry73204/cavedu-intelligent-house
%\end{rSubsection}


\begin{rSubsection}{Lab of Algorithmic Research -- Undergraduate Researcher}{01/2014 - 06/2016}{}{}%{January 2014 - June 2016}{Undergraduate Researcher}{Supervisor: Prof.Hsueh-I Lu}
\item Multiple source terminal pair for non-crossing Shortest Paths on planear graphs.
\item Combined techniques of dense distanse graphs and graph decomposition to achive lower complexity.
\end{rSubsection}
%{
%\begin{rSubsection}{Software Developer}{September 2014 - October 2014}{NTU %Laboratory for Language Technologies}{Supervisor: Prof.Zhao-Ming Gao}
%\item Natural Language Processing.
%\item Python, PHP, R scripting with MySQL db.
%\end{rSubsection}

\end{rSection}

\begin{rSection}{Projects}

\begin{rSubsection}{Machine Learning -- Temporal Point Process}{05/2017}{}{}
%\item The Network Hawkes Process predicts the relations between Temporal Point Process. \\Previous approaches used Bayesian models to reconstruct the underlying network. 
\item Solves the Network Hawkes problem with Density estimation, Markov chains, and Deep Learning.
\item Implementations: Two layer Neural Network, Convolutional logistic regression.
\item Accuracy of the edge prediction were 85\%. Recovered response functions without any priors.
%\item https://github.com/ryucc/ShallowHawkes/
\end{rSubsection}


\begin{rSubsection}{Computer Vision -- Hyper-lapse Video Stabilization}{05/2017}{}{}
%\item{} [Joshi, Neel, et al 2015] used the sum-of-keypoint distance as a minimization target for frame selection. We formulated the jittering effect to as measure, proving a new frame selection algorithm. 
\item Proposed to use the amplitude of jittering as a optimization target for frame selection.
\item Results gave human eye more comfort, Optical Flow algorithms better optimization. 
\item https://sites.google.com/view/cs766-hyperlapse-video/
\end{rSubsection}

%\begin{rSubsection}{Computer Vision -- Medical Image Analysis}{11/2016}{}{}
%\item Implemented divergence metrics for the C++ ITK open source imaging toolkit.
%\item MRI/CT Brain image resigtration.
%\item http://pages.cs.wisc.edu/\textasciitilde kurtis/ITKMetrics/
%\end{rSubsection}

\begin{rSubsection}{Artificial Intelligence -- Movie Clip Catagorization}{05/2015}{}{}
\item A Project to classify a set of movie clips in labeled in 3 catagories. 
\item K-Nearest Neighbors, Bag of words, Support vector machines.
\item SIFT features, Image Segmentation, Optical flow features. Fourier Transforms.
\item Cross Validation showed a 80\% accuracy on the data set.

\end{rSubsection}

\end{rSection}

%------------------------------------------------




%\begin{rSection}{Research Experience}

%\begin{rSubsection}{Lab of Algorithmic Research}{January 2014 - June 2016}{Undergraduate Researcher}{Supervisor: Prof.Hsueh-I Lu}
%\item Multiple s-t pair Non-crossing Shortest Paths
%\end{rSubsection}
%\end{rSection}

%------------------------------------------------

%----------------------------------------------------------------------------------------
%	SKILLS SECTION
%----------------------------------------------------------------------------------------

\begin{rSection}{Technical Skills}
Programming languages: C++, C, Python, Matlab, Shell Scripting\\
Platforms: Linux, Git  \\
Toolkits: TensorFlow, Keras, OpenCV, OpenCL\\

\end{rSection}

%------------------------------------------------
%----------------------------------------------------------------------------------------
%	EXAMPLE SECTION
%----------------------------------------------------------------------------------------

%\begin{rSection}{Section Name}

%Section content\ldots

%\end{rSection}

%----------------------------------------------------------------------------------------

\end{document}
